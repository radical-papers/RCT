\newif\ifdraft{}
% \drafttrue{}

\ifdraft{}
  \newcommand{\amnote}[1]{ \textcolor{blue} { ***andrem: #1 }}
  \newcommand{\jhanote}[1]{ {\textcolor{red} { ***shantenu: #1 }}}
  \newcommand{\mtnote}[1]{ {\textcolor{orange} { ***matteo: #1 }}}
\else
  \newcommand{\amnote}[1]{}
  \newcommand{\jhanote}[1]{}
  \newcommand{\mtnote}[1]{}
\fi

\newcommand{\apples}{AppLeS\xspace}
\newcommand{\bj}{BigJob\xspace}
\newcommand{\computeunit}{Compute-Unit\xspace}
\newcommand{\computeunits}{Compute-Units\xspace}
\newcommand{\cloud}{cloud\xspace}
\newcommand{\clouds}{clouds\xspace}
\newcommand{\cc}{c\&c\xspace}
\newcommand{\CC}{C\&C\xspace}
\newcommand{\computedataservice}{Compute-Data Service\xspace}
\newcommand{\cu}{CU\xspace}
\newcommand{\cus}{CUs\xspace}
\newcommand{\dataunit}{Data-Unit\xspace}
\newcommand{\dataunits}{Data-Units\xspace}
\newcommand{\du}{DU\xspace}
\newcommand{\dus}{DUs\xspace}
\newcommand{\mrmg}{MR-Manager\xspace}
\newcommand{\MW}{master-worker\xspace}
\newcommand{\numrep}{8 }
\newcommand{\panda}{PanDA\xspace}
\newcommand{\pilot}{Pilot\xspace}
\newcommand{\pilots}{Pilots\xspace}
\newcommand{\pilotjob}{Pilot-Job\xspace}
\newcommand{\pilotjobs}{Pilot-Jobs\xspace}
\newcommand{\pilotcompute}{Pilot-Compute\xspace}
\newcommand{\pilotcomputes}{Pilot-Computes\xspace}
\newcommand{\pilotdata}{Pilot-Data\xspace}
\newcommand{\pilotdataservice}{Pilot-Data Service\xspace}
\newcommand{\pilotcomputeservice}{Pilot-Compute Service\xspace}
\newcommand{\prop}[1]{\textit{#1}\xspace}
\newcommand{\pilotmapreduce}{PilotMapReduce\xspace}
\newcommand{\pstar}{P*\xspace}
\newcommand{\pd}{PD\xspace}
\newcommand{\pj}{PJ\xspace}
\newcommand{\pjs}{PJs\xspace}
\newcommand{\pds}{Pilot Data Service\xspace}
\newcommand{\samplenum}{4 }
\newcommand{\su}{SU\xspace}
\newcommand{\sus}{SUs\xspace}
\newcommand{\schedulableunit}{Schedulable Unit\xspace}
\newcommand{\schedulableunits}{Schedulable Units\xspace}
\newcommand{\tmax}{\(T_{max}\)}
\newcommand{\tc}{\(T_{C}\)}
\newcommand{\tcnsp}{\(T_{C}\)}
\newcommand{\vocab}[1]{\textbf{#1}\xspace}

\newcommand{\B}[1]{\textbf{#1}\xspace}
\newcommand{\C}[1]{\textsc{#1}\xspace}
\newcommand{\F}[1]{\textbf{FIXME\@: #1}\xspace}
\newcommand{\I}[1]{\textit{#1}\xspace}
\newcommand{\T}[1]{\texttt{#1}\xspace}

\newcommand{\impterm}[1]{\texttt{#1}\xspace}

% System names
\newcommand{\bw}{\I{Blue\,Waters}}
\newcommand{\stampede}{\I{Stampede}}
\newcommand{\comet}{\I{Comet}}
\newcommand{\titan}{\I{Titan}}

% Latex Fu
\newcommand{\UPP}{\vspace*{-2.0em}}
\newcommand{\UP}{\vspace*{-1.0em}}
\newcommand{\up}{\vspace*{-0.5em}}

% Paper specific Macro's
\newcommand{\ru}{$RU$\xspace}
\newcommand{\ttc}{$ttc$}
\newcommand{\ttca}{$ttc_a$}

% Table multirows
\newcommand{\mr}[1]{\multirow{2}{*}{#1}}%
\newcommand{\mc}[2]{\multicolumn{#1}{l}{#2}}

\lstdefinestyle{myListing}{
  frame=single,   
  backgroundcolor=\color{listinggray},  
  %float=t,
  language=C,       
  basicstyle=\ttfamily \footnotesize,
  breakautoindent=true,
  breaklines=true
  tabsize=2,
  captionpos=b,  
  aboveskip=0em,
  belowskip=-2em,
}      

\lstdefinestyle{myPythonListing}{
  frame=single,   
  backgroundcolor=\color{listinggray},  
  %float=t,
  language=Python,       
  basicstyle=\ttfamily \footnotesize,
  breakautoindent=true,
  breaklines=true
  tabsize=2,
  captionpos=b,  
}

% This is now the recommended way for checking for PDFLaTeX:
\ifpdf{}
  \DeclareGraphicsExtensions{.pdf, .jpg, .tif}
\else
  \DeclareGraphicsExtensions{.eps, .jpg, .ps}
\fi

\tolerance=1000
\hyphenpenalty=10
