\documentclass[preprint,12pt, a4paper]{elsarticle}

\usepackage{amsmath}
\usepackage{amssymb}
\usepackage{array}
\usepackage{booktabs}
\usepackage{color}
\usepackage{float}
\usepackage{graphicx}
\usepackage{ifpdf}
\usepackage[utf8]{inputenc}
\usepackage{keyval}
\usepackage{lineno}
\usepackage{listings}
\usepackage{longtable}
\usepackage{moresize}
\usepackage{multirow}
\usepackage{paralist}
\usepackage{rotating}
\usepackage{soul}
\usepackage{srcltx}
\usepackage{url}
\usepackage{xcolor}
\usepackage{xspace}
\usepackage{wrapfig}
\usepackage{subfig}
\usepackage{tikz}
\usepackage[export]{adjustbox} % http://ctan.org/pkg/adjustbox

\definecolor{listinggray}{gray}{0.95}
\definecolor{darkgray}{gray}{0.7}
\definecolor{commentgreen}{rgb}{0, 0.4, 0}
\definecolor{darkblue}{rgb}{0, 0, 0.6}
\definecolor{purple}{rgb}{0.6, 0, 0.6}
\definecolor{middleblue}{rgb}{0, 0, 0.75}
\definecolor{darkred}{rgb}{0.4, 0, 0}
\definecolor{brown}{rgb}{0.5, 0.5, 0}
\definecolor{dkgreen}{rgb}{0,0.5,0}
\definecolor{orange}{rgb}{1,.5,0}
\definecolor{dandelion}{cmyk}{0,0.29,0.84,0}

\usepackage[normalem]{ulem}
\makeatletter
\def\cyanuwave{\bgroup \markoverwith{\lower3.5\p@\hbox{\sixly \textcolor{cyan}{\char58}}}\ULon}
\def\reduwave{\bgroup \markoverwith{\lower3.5\p@\hbox{\sixly \textcolor{red}{\char58}}}\ULon}
\def\blueuwave{\bgroup \markoverwith{\lower3.5\p@\hbox{\sixly \textcolor{blue}{\char58}}}\ULon}
\font\sixly=lasy6 % does not re-load if already loaded, so no memory problem.
\makeatother

\def\BibTeX{{\rm B\kern-.05em{\sc i\kern-.025em b}\kern-.08em
    T\kern-.1667em\lower.7ex\hbox{E}\kern-.125emX}}

% Generate circled numbers
\newcommand*\circled[1]{\tikz[baseline=(char.base)]{
    \node[shape=circle,draw,inner sep=1pt] (char) {#1};}}
\newif\ifdraft{}
%\drafttrue{}

\ifdraft{}
  \newcommand{\amnote}[1]{ \textcolor{blue} { ***andrem: #1 }}
  \newcommand{\jhanote}[1]{ {\textcolor{red} { ***shantenu: #1 }}}
  \newcommand{\mtnote}[1]{ {\textcolor{orange} { ***matteo: #1 }}}
\else
  \newcommand{\amnote}[1]{}
  \newcommand{\jhanote}[1]{}
  \newcommand{\mtnote}[1]{}
\fi

\newcommand{\apples}{AppLeS\xspace}
\newcommand{\bj}{BigJob\xspace}
\newcommand{\computeunit}{Compute-Unit\xspace}
\newcommand{\computeunits}{Compute-Units\xspace}
\newcommand{\cloud}{cloud\xspace}
\newcommand{\clouds}{clouds\xspace}
\newcommand{\cc}{c\&c\xspace}
\newcommand{\CC}{C\&C\xspace}
\newcommand{\computedataservice}{Compute-Data Service\xspace}
\newcommand{\cu}{CU\xspace}
\newcommand{\cus}{CUs\xspace}
\newcommand{\dataunit}{Data-Unit\xspace}
\newcommand{\dataunits}{Data-Units\xspace}
\newcommand{\du}{DU\xspace}
\newcommand{\dus}{DUs\xspace}
\newcommand{\mrmg}{MR-Manager\xspace}
\newcommand{\MW}{master-worker\xspace}
\newcommand{\numrep}{8 }
\newcommand{\panda}{PanDA\xspace}
\newcommand{\pilot}{Pilot\xspace}
\newcommand{\pilots}{Pilots\xspace}
\newcommand{\pilotjob}{Pilot-Job\xspace}
\newcommand{\pilotjobs}{Pilot-Jobs\xspace}
\newcommand{\pilotcompute}{Pilot-Compute\xspace}
\newcommand{\pilotcomputes}{Pilot-Computes\xspace}
\newcommand{\pilotdata}{Pilot-Data\xspace}
\newcommand{\pilotdataservice}{Pilot-Data Service\xspace}
\newcommand{\pilotcomputeservice}{Pilot-Compute Service\xspace}
\newcommand{\prop}[1]{\textit{#1}\xspace}
\newcommand{\pilotmapreduce}{PilotMapReduce\xspace}
\newcommand{\pstar}{P*\xspace}
\newcommand{\pd}{PD\xspace}
\newcommand{\pj}{PJ\xspace}
\newcommand{\pjs}{PJs\xspace}
\newcommand{\pds}{Pilot Data Service\xspace}
\newcommand{\samplenum}{4 }
\newcommand{\su}{SU\xspace}
\newcommand{\sus}{SUs\xspace}
\newcommand{\schedulableunit}{Schedulable Unit\xspace}
\newcommand{\schedulableunits}{Schedulable Units\xspace}
\newcommand{\tmax}{\(T_{max}\)}
\newcommand{\tc}{\(T_{C}\)}
\newcommand{\tcnsp}{\(T_{C}\)}
\newcommand{\vocab}[1]{\textbf{#1}\xspace}

\newcommand{\B}[1]{\textbf{#1}\xspace}
\newcommand{\C}[1]{\textsc{#1}\xspace}
\newcommand{\F}[1]{\textbf{FIXME\@: #1}\xspace}
\newcommand{\I}[1]{\textit{#1}\xspace}
\newcommand{\T}[1]{\texttt{#1}\xspace}

\newcommand{\impterm}[1]{\texttt{#1}\xspace}

% System names
\newcommand{\bw}{\I{Blue\,Waters}}
\newcommand{\stampede}{\I{Stampede}}
\newcommand{\comet}{\I{Comet}}
\newcommand{\titan}{\I{Titan}}

% Latex Fu
\newcommand{\UPP}{\vspace*{-2.0em}}
\newcommand{\UP}{\vspace*{-1.0em}}
\newcommand{\up}{\vspace*{-0.5em}}

% Paper specific Macro's
\newcommand{\ru}{$RU$\xspace}
\newcommand{\ttc}{$ttc$}
\newcommand{\ttca}{$ttc_a$}

% Table multirows
\newcommand{\mr}[1]{\multirow{2}{*}{#1}}%
\newcommand{\mc}[2]{\multicolumn{#1}{l}{#2}}

\lstdefinestyle{myListing}{
  frame=single,   
  backgroundcolor=\color{listinggray},  
  %float=t,
  language=C,       
  basicstyle=\ttfamily \footnotesize,
  breakautoindent=true,
  breaklines=true
  tabsize=2,
  captionpos=b,  
  aboveskip=0em,
  belowskip=-2em,
}      

\lstdefinestyle{myPythonListing}{
  frame=single,   
  backgroundcolor=\color{listinggray},  
  %float=t,
  language=Python,       
  basicstyle=\ttfamily \footnotesize,
  breakautoindent=true,
  breaklines=true
  tabsize=2,
  captionpos=b,  
}

% This is now the recommended way for checking for PDFLaTeX:
\ifpdf{}
  \DeclareGraphicsExtensions{.pdf, .jpg, .tif}
\else
  \DeclareGraphicsExtensions{.eps, .jpg, .ps}
\fi

\tolerance=1000
\hyphenpenalty=10


\journal{SoftwareX}

\begin{document}
\begin{frontmatter}

% \title{Title/Name of your software}
\title{RADICAL Cybertools}

\author{A. Author}
\address{Your institute, some address}

\author{B. Author}
\address{Your institute, some address}

\author{C. Author}
\address{Your institute, some address}

\author{D. Author}
\address{Your institute, some address}

\begin{abstract}
Ca. 100 words
\end{abstract}

\begin{keyword}
%% keywords here, in the form: keyword \sep keyword
keyword 1 \sep keyword 2 \sep keyword 3

%% PACS codes here, in the form: \PACS code \sep code

%% MSC codes here, in the form: \MSC code \sep code
%% or \MSC[2008] code \sep code (2000 is the default)

\end{keyword}

\end{frontmatter}

\linenumbers{}

% The manuscript must be submitted in single column. The following
% constraints apply: 
% Word count: max. 3000.
% a.	Excluding: title, authors, affiliations, references, metadata tables.
% b.	Including: abstract, running text, captions, footnotes.
% c.	Max. 6 figures.


% ---------------------------------------------------------------------------
% Section I
% ---------------------------------------------------------------------------
\section{Motivation and significance}\label{sec:motivation}

{\em Guidelines for the authors:
\begin{enumerate}
	\item Introduce the scientific background and the motivation for
		    developing the software.
  \item Explain why the software is important, and describe the exact
        (scientific) problem(s) it solves.
	\item Indicate in what way the software has contributed (or how it will
        contribute in the future) to the process of scientific discovery;
        if available, this is to be supported by citing a research paper
        using the software.
  \item Provide a description of the experimental setting (how does the 
        user use the software?).
  \item Introduce related work in literature (cite or list algorithms 
          used, other software etc.).
\end{enumerate}}

The RADICAL Cyberinfrastructure tools (RCT)

\begin{itemize}
  \item Background:
    \begin{itemize}
      \item Growing importance of multi-task applications in multiple
      research fields.
      \item Need for scalability across several dimensions and
      infrastructures.
      \item Need for advanced capabilities like adaptivity, streaming,
      fault-tolerance, etc.
      \item The importance of supporting execution of [homo|hetero]geneous
      multi-task applications on HPC infrastructures.
      \item The rise of the pilot paradigm.
    \end{itemize}
  \item Motivations:
    \begin{itemize}
      \item Multi-task applications can be encoded into a variety of
      workflows and be executed on a variety of infrastructures.
      \item Users require flexible ways to encode workflows without having to
      buy into a specific logical representation or concrete implementation
      language.
      \item Separation between workflow description, workflow management,
      resource acquisition, workflow execution.
      \item End-to-end, monolithic and encompassing solutions do not satisfy
      these requirements.
      \item Need for an ecosystem of software modules that can be composed
      and configured depending on specific use cases.
      \item Need for rapid development of domain-specific workflow engines
      capable of supporting specific requirements without having to implement
      capabilities for resource acquisition and execution management.
      \item RCT satisfy these requirements.
    \end{itemize}
  \item Scientific relevance:
    \begin{itemize}
      \item Earth science: brief description and references
      \item Climate science: brief description and references
      \item Molecular biology: brief description and references
      \item Polar science: brief description and references
      \item \ldots
    \end{itemize}
  \item Settings:
    \begin{itemize}
      \item Python libraries supporting several deployment scenarios: local,
      local/remote, remote, multiple remote.
    \end{itemize}
  \item Related work:
    \begin{itemize}
      \item Middleware for scientific computing. Derive from: RP paper, Pilot
      paper, EnTK papers.
    \end{itemize}
\end{itemize}


% ---------------------------------------------------------------------------
% Section II
% ---------------------------------------------------------------------------
\section{Software description}\label{sec:description}

{\em Describe the software in as much as is necessary to establish a
vocabulary needed to explain its impact.}

\begin{itemize}
  \item Currently, RTC implements four types of software system: (i)
  workflow management system; (ii) pilot system; (iii) interoperability
  interface; and (iv) profiler.
  \item Building block design methodology
  \item open source development model (github, tickets, documentation, etc.)
\end{itemize}


% ---------------------------------------------------------------------------
% Section III
% ---------------------------------------------------------------------------
\subsection{Software Architecture}\label{sec:architecture}

{\em Give a short overview of the overall software architecture; provide a
pictorial component overview or similar (if possible). If necessary provide
implementation details.}

\begin{itemize}
  \item Create an overall architecture diagram for the whole RCT\@: EnTK, RP,
  RS, RA\@. Compose this diagram from the existing diagram for each RCT
  component, making it visually uniform.
\end{itemize}


% ---------------------------------------------------------------------------
% Section IV
% ---------------------------------------------------------------------------
\subsection{Software Functionalities}\label{sec:functionalities}

{\em Present the major functionalities of the software.}

\begin{itemize}
  \item EnTK\@:
  \begin{itemize}
    \item Functionalities (VB)
  \end{itemize}
  \item RP\@: 
  \begin{itemize}
    \item Functionalities (AM)
  \end{itemize}
  \item RS\@:
  \begin{itemize}
    \item Functionalities (AM)
  \end{itemize}
  \item RA\@:)
  \begin{itemize}
    \item Functionalities (MT)
  \end{itemize}
\end{itemize}


% ---------------------------------------------------------------------------
% Section V
% ---------------------------------------------------------------------------
\subsection{Sample code snippets analysis (optional)}\label{sec:code}

\begin{itemize}
  \item EnTK PST API
  \item RP pilot and unit API
  \item SAGA API
  \item RA API
\end{itemize}

% ---------------------------------------------------------------------------
% Section VI
% ---------------------------------------------------------------------------
\section{Illustrative Examples}\label{sec:examples}

{\em Provide at least one illustrative example to demonstrate the major
functions.

Optional: you may include one explanatory video that will appear next to your
article, in the right hand side panel. (Please upload any video as a single
supplementary file with your article. Only one MP4 formatted, with 50MB
maximum size, video is possible per article. Recommended video dimensions are
640 x 480 at a maximum of 30 frames/second. Prior to submission please test
and validate your .mp4 file at $
http://elsevier-apps.sciverse.com/GadgetVideoPodcastPlayerWeb/verification$.
This tool will display your video exactly in the same way as it will appear
on ScienceDirect.).}

\begin{itemize}
  \item Simple EnTK application or one of the domain-specific workflows
  (HTBAC?) with RA analysis?
  \item Video of the execution, why not.
\end{itemize}

% ---------------------------------------------------------------------------
% Section VII
% ---------------------------------------------------------------------------
\section{Impact}\label{sec:impact}

{\em \textbf{This is the main section of the article and the reviewers weight
the description here appropriately}

Guidelines for the authors:
\begin{enumerate}
  \item Indicate in what way new research questions can be pursued as a
  result of the software (if any).
  \item Indicate in what way, and to what extent, the pursuit of existing
  research questions is improved (if so).
  \item Indicate in what way the software has changed the daily practice of
  its users (if so).
  \item Indicate how widespread the use of the software is within and outside
  the intended user group.
  \item Indicate in what way the software is used in commercial settings
  and/or how it led to the creation of spin-off companies (if so).
\end{enumerate}}

\begin{itemize}
  \item Support of new research questions via RCT\@:
  \begin{itemize}
    \item General purpose pilot API, independent from domain, project,
    infrastructure or middleware specific requirements.
    \item General purpose task execution model: heterogeneous and.or
    homogeneous set of tasks distributed across multiple
    infrastructures/machines or distributed within a single machine. No
    requirement for a specific distribution or communication pattern across
    tasks.
  \end{itemize}
  \item Improved support of existing research question via RCT\@:
  \begin{itemize}
    \item Rapid codification of domain-specific applications due to isolation
    of concerns between application logic and resource and execution
    management.
    \item Faster and simpler programmatic specification of tasks dependences
    with the PST model.
    \item User can control trade off between concurrency and duration of the
    execution of multi-task applications (via pilots).
    \item Strong and weak scaling on multiple HPC machines.
  \end{itemize}
  \item Impact on user practices:
  \begin{itemize}
    \item Writing domain-specific applications without having to
    code/configure resource acquisition and execution management.
    \item Rapid implementation of adaptive runtime strategies within
    domain-specific applications.
    \item Different modalities of deployment, including working for the user
    workstation (or VM) without having to log into the computing
    infrastructure.
    \item Support of a large set of HPC machines via a unified and consistent
    interface.
  \end{itemize}
  \item Adoption figures:
  \begin{itemize}
    \item Aggregated numbers for all RCT.
    \item Break down figures?
  \end{itemize}
\end{itemize}


% ---------------------------------------------------------------------------
% Section VIII
% ---------------------------------------------------------------------------
\section{Conclusions}\label{sec:conclusions}

{\em Set out the conclusion of this original software publication.}


% ---------------------------------------------------------------------------
% Acknowledgements
% ---------------------------------------------------------------------------
\section*{Acknowledgements}

{\em Optionally thank people and institutes you need to acknowledge. }

%% The Appendices part is started with the command \appendix;
%% appendix sections are then done as normal sections
%% \appendix

%% \section{}
%% \label{}


% ---------------------------------------------------------------------------
% References
% ---------------------------------------------------------------------------
\bibliographystyle{elsarticle-num} 
\bibliography{rct-software}


% ---------------------------------------------------------------------------
% Appendix I
% ---------------------------------------------------------------------------
\section*{Required Metadata}\label{sec:metadata}


% ---------------------------------------------------------------------------
% Appendix II
% ---------------------------------------------------------------------------
\section*{Current code version}\label{sec:src_version}

{\em Ancillary data table required for subversion of the codebase. Kindly
replace examples in right column with the correct information about your
current code, and leave the left column as it is.

\begin{table}[!ht]
\begin{tabular}{|l|p{6.5cm}|p{6.5cm}|}
\hline
\textbf{Nr.}                                                     & 
\textbf{Code metadata description}                               & 
\textbf{Please fill in this column}                              \\
\hline
C1                                                               & 
Current code version                                             & 
For example v42                                                  \\
\hline
C2                                                               & 
Permanent link to code/repository used for this code version     & 
For example: $https://github.com/mozart/mozart2$                 \\
\hline
C3                                                               & 
Legal Code License                                               & 
List one of the approved licenses                                \\
\hline
C4                                                               & 
Code versioning system used                                      & 
For example svn, git, mercurial, etc. put none if none           \\
\hline
C5                                                               & 
Software code languages, tools, and services used                & 
For example C++, python, r, MPI, OpenCL, etc.                    \\
\hline
C6                                                               & 
Compilation requirements, operating environments \& dependencies & 
                                                                 \\
\hline
C7                                                               & 
If available Link to developer documentation/manual              & 
For example: $http://mozart.github.io/documentation/$            \\
\hline
C8                                                               & 
Support email for questions                                      & 
                                                                 \\
\hline
\end{tabular}
\caption{Code metadata (mandatory)}\label{tab:src_metadata} 
\end{table}}


% ---------------------------------------------------------------------------
% Appendix III
% ---------------------------------------------------------------------------
\section*{Current executable software version}\label{sec:bin_version}

{\em Ancillary data table required for sub version of the executable
software: (x.1, x.2 etc.) kindly replace examples in right column with the
correct information about your executables, and leave the left column as it
is.

\begin{table}[!ht]
\begin{tabular}{|l|p{6.5cm}|p{6.5cm}|}
\hline
\textbf{Nr.} & 
\textbf{(Executable) software metadata description} & 
\textbf{Please fill in this column} \\
\hline
S1 & 
Current software version & 
For example 1.1, 2.4 etc. \\
\hline
S2 & 
Permanent link to executables of this version  & 
For example: $https://github.com/combogenomics/$ $DuctApe/releases/tag/DuctApe-0.16.4$ 
\\
\hline
S3 & 
Legal Software License & 
List one of the approved licenses \\
\hline
S4 & 
Computing platforms/Operating Systems & 
For example Android, BSD, iOS, Linux, OS X, Microsoft Windows, Unix-like , IBM z/OS, distributed/web based etc. \\
\hline
S5 & 
Installation requirements \& dependencies & 
\\
\hline
S6 & 
If available, link to user manual - if formally published include a reference to the publication in the reference list & 
For example: $http://mozart.github.io/documentation/$ \\
\hline
S7 & 
Support email for questions & 
\\
\hline
\end{tabular}
\caption{Software metadata (optional)}
\label{tab:bin_metadata} 
\end{table}}

\end{document}
\endinput