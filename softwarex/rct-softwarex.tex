\documentclass[preprint,12pt, a4paper]{elsarticle}

\usepackage{amsmath}
\usepackage{amssymb}
\usepackage{array}
\usepackage{booktabs}
\usepackage{color}
\usepackage{float}
\usepackage{graphicx}
\usepackage{ifpdf}
\usepackage[utf8]{inputenc}
\usepackage{keyval}
\usepackage{lineno}
\usepackage{listings}
\usepackage{longtable}
\usepackage{moresize}
\usepackage{multirow}
\usepackage{paralist}
\usepackage{rotating}
\usepackage{soul}
\usepackage{srcltx}
\usepackage{url}
\usepackage{xcolor}
\usepackage{xspace}
\usepackage{wrapfig}
\usepackage{subfig}
\usepackage{tikz}
\usepackage[export]{adjustbox} % http://ctan.org/pkg/adjustbox

\definecolor{listinggray}{gray}{0.95}
\definecolor{darkgray}{gray}{0.7}
\definecolor{commentgreen}{rgb}{0, 0.4, 0}
\definecolor{darkblue}{rgb}{0, 0, 0.6}
\definecolor{purple}{rgb}{0.6, 0, 0.6}
\definecolor{middleblue}{rgb}{0, 0, 0.75}
\definecolor{darkred}{rgb}{0.4, 0, 0}
\definecolor{brown}{rgb}{0.5, 0.5, 0}
\definecolor{dkgreen}{rgb}{0,0.5,0}
\definecolor{orange}{rgb}{1,.5,0}
\definecolor{dandelion}{cmyk}{0,0.29,0.84,0}

\usepackage[normalem]{ulem}
\makeatletter
\def\cyanuwave{\bgroup \markoverwith{\lower3.5\p@\hbox{\sixly \textcolor{cyan}{\char58}}}\ULon}
\def\reduwave{\bgroup \markoverwith{\lower3.5\p@\hbox{\sixly \textcolor{red}{\char58}}}\ULon}
\def\blueuwave{\bgroup \markoverwith{\lower3.5\p@\hbox{\sixly \textcolor{blue}{\char58}}}\ULon}
\font\sixly=lasy6 % does not re-load if already loaded, so no memory problem.
\makeatother

\def\BibTeX{{\rm B\kern-.05em{\sc i\kern-.025em b}\kern-.08em
    T\kern-.1667em\lower.7ex\hbox{E}\kern-.125emX}}

% Generate circled numbers
\newcommand*\circled[1]{\tikz[baseline=(char.base)]{
    \node[shape=circle,draw,inner sep=1pt] (char) {#1};}}
\newif\ifdraft{}
\drafttrue{}

\ifdraft{}
  \newcommand{\amnote}[1]{ \textcolor{blue} { ***andrem: #1 }}
  \newcommand{\jhanote}[1]{ {\textcolor{red} { ***shantenu: #1 }}}
  \newcommand{\mtnote}[1]{ {\textcolor{orange} { ***matteo: #1 }}}
\else
  \newcommand{\amnote}[1]{}
  \newcommand{\jhanote}[1]{}
  \newcommand{\mtnote}[1]{}
\fi

\newcommand{\apples}{AppLeS\xspace}
\newcommand{\bj}{BigJob\xspace}
\newcommand{\computeunit}{Compute-Unit\xspace}
\newcommand{\computeunits}{Compute-Units\xspace}
\newcommand{\cloud}{cloud\xspace}
\newcommand{\clouds}{clouds\xspace}
\newcommand{\cc}{c\&c\xspace}
\newcommand{\CC}{C\&C\xspace}
\newcommand{\computedataservice}{Compute-Data Service\xspace}
\newcommand{\cu}{CU\xspace}
\newcommand{\cus}{CUs\xspace}
\newcommand{\dataunit}{Data-Unit\xspace}
\newcommand{\dataunits}{Data-Units\xspace}
\newcommand{\du}{DU\xspace}
\newcommand{\dus}{DUs\xspace}
\newcommand{\mrmg}{MR-Manager\xspace}
\newcommand{\MW}{master-worker\xspace}
\newcommand{\numrep}{8 }
\newcommand{\panda}{PanDA\xspace}
\newcommand{\pilot}{Pilot\xspace}
\newcommand{\pilots}{Pilots\xspace}
\newcommand{\pilotjob}{Pilot-Job\xspace}
\newcommand{\pilotjobs}{Pilot-Jobs\xspace}
\newcommand{\pilotcompute}{Pilot-Compute\xspace}
\newcommand{\pilotcomputes}{Pilot-Computes\xspace}
\newcommand{\pilotdata}{Pilot-Data\xspace}
\newcommand{\pilotdataservice}{Pilot-Data Service\xspace}
\newcommand{\pilotcomputeservice}{Pilot-Compute Service\xspace}
\newcommand{\prop}[1]{\textit{#1}\xspace}
\newcommand{\pilotmapreduce}{PilotMapReduce\xspace}
\newcommand{\pstar}{P*\xspace}
\newcommand{\pd}{PD\xspace}
\newcommand{\pj}{PJ\xspace}
\newcommand{\pjs}{PJs\xspace}
\newcommand{\pds}{Pilot Data Service\xspace}
\newcommand{\samplenum}{4 }
\newcommand{\su}{SU\xspace}
\newcommand{\sus}{SUs\xspace}
\newcommand{\schedulableunit}{Schedulable Unit\xspace}
\newcommand{\schedulableunits}{Schedulable Units\xspace}
\newcommand{\tmax}{\(T_{max}\)}
\newcommand{\tc}{\(T_{C}\)}
\newcommand{\tcnsp}{\(T_{C}\)}
\newcommand{\vocab}[1]{\textbf{#1}\xspace}

\newcommand{\B}[1]{\textbf{#1}\xspace}
\newcommand{\C}[1]{\textsc{#1}\xspace}
\newcommand{\F}[1]{\textbf{FIXME\@: #1}\xspace}
\newcommand{\I}[1]{\textit{#1}\xspace}
\newcommand{\T}[1]{\texttt{#1}\xspace}

\newcommand{\impterm}[1]{\texttt{#1}\xspace}

% System names
\newcommand{\bw}{\I{Blue\,Waters}}
\newcommand{\stampede}{\I{Stampede}}
\newcommand{\comet}{\I{Comet}}
\newcommand{\titan}{\I{Titan}}

% Latex Fu
\newcommand{\UPP}{\vspace*{-2.0em}}
\newcommand{\UP}{\vspace*{-1.0em}}
\newcommand{\up}{\vspace*{-0.5em}}

% Paper specific Macro's
\newcommand{\ru}{$RU$\xspace}
\newcommand{\ttc}{$ttc$}
\newcommand{\ttca}{$ttc_a$}

% Table multirows
\newcommand{\mr}[1]{\multirow{2}{*}{#1}}%
\newcommand{\mc}[2]{\multicolumn{#1}{l}{#2}}

\lstdefinestyle{myListing}{
  frame=single,   
  backgroundcolor=\color{listinggray},  
  %float=t,
  language=C,       
  basicstyle=\ttfamily \footnotesize,
  breakautoindent=true,
  breaklines=true
  tabsize=2,
  captionpos=b,  
  aboveskip=0em,
  belowskip=-2em,
}      

\lstdefinestyle{myPythonListing}{
  frame=single,   
  backgroundcolor=\color{listinggray},  
  %float=t,
  language=Python,       
  basicstyle=\ttfamily \footnotesize,
  breakautoindent=true,
  breaklines=true
  tabsize=2,
  captionpos=b,  
}

% This is now the recommended way for checking for PDFLaTeX:
\ifpdf{}
  \DeclareGraphicsExtensions{.pdf, .jpg, .tif}
\else
  \DeclareGraphicsExtensions{.eps, .jpg, .ps}
\fi

\tolerance=1000
\hyphenpenalty=10


\journal{SoftwareX}

\begin{document}
\begin{frontmatter}

% \title{Title/Name of your software}
\title{RADICAL Cybertools}

\author{A. Author}
\address{Your institute, some address}

\author{B. Author}
\address{Your institute, some address}

\author{C. Author}
\address{Your institute, some address}

\author{D. Author}
\address{Your institute, some address}

\begin{abstract}
% Ca. 100 words
RADICAL-Cybertools (RCT) are a set of software systems developed to support
efficient and effective scientific computing. RCT enable executing many-task
applications at extreme scale and on a variety of computing infrastructures,
including the leadership-class machines managed by ORNL, NCSA and TACC. RCT
allow rapid development of applications for software engineering, chemical
physics, materials science, health science, climate science, drug discovery
and particle physics. RCT are building blocks, designed to work as
stand-alone systems, integrated among themselves or integrated with
third-party systems. Between 2013 and 2019, RCT enabled innovative science
consuming hundreds of millions of core hours.
\end{abstract}

\begin{keyword}
%% keywords here, in the form: keyword \sep keyword
Middleware \sep Pilot System \sep Workflow Engine

%% PACS codes here, in the form: \PACS code \sep code

%% MSC codes here, in the form: \MSC code \sep code
%% or \MSC[2008] code \sep code (2000 is the default)

\end{keyword}

\end{frontmatter}

\linenumbers{}

% {\em The manuscript must be submitted in single column. The following
% constraints apply:
% \begin{enumerate}
%   \item Word count: max. 3000. Excluding: title, authors, affiliations,
%         references, metadata tables; Including: abstract, running text,
%         captions, footnotes.
%   \item Max. 6 figures.
%   \item The manuscript must have line numbers.
%   \item At least one of the two mandatory metadata tables located at the
%         end of the article template must be filled in and included in the
%         manuscript.
%   \item To submit your code/software you can: Point to an external location
%         (repository, archive, etc.) where the code is publicly available.
%         Clearly mark the address to retrieve the code in the metadata
%         tables included in the manuscript.
% \end{enumerate}}


% ---------------------------------------------------------------------------
% Section I
% ---------------------------------------------------------------------------
\section{Motivation and significance}\label{sec:motivation}

% {\em Guidelines for the authors:
% \begin{enumerate}
%   \item Introduce the scientific background and the motivation for
%         developing the software.
%   \item Explain why the software is important, and describe the exact
%         (scientific) problem(s) it solves.
%   \item Indicate in what way the software has contributed (or how it will
%         contribute in the future) to the process of scientific discovery;
%         if available, this is to be supported by citing a research paper
%         using the software.
%   \item Provide a description of the experimental setting (how does the
%         user use the software?).
%   \item Introduce related work in literature (cite or list algorithms used,
%           other software etc.).
% \end{enumerate}}

The design of distributed systems to support scientific computing has never
been more challenging. Unprecedented diversity in application requirements,
and disruptive changes in the resources and technology landscapes, intermix
with new discovery modalities and need for scalable computing.

Set against this dynamic landscape, two critical question must be addressed:
How can middleware be designed and implemented to meet the collective
challenges of scale, new and diverse functionality, and usability? How can
critical middleware components be designed to be sustainable software
implementations while being forward looking and enable innovative
capabilities?

RADICAL-Cybertools (RCT) are a set of systems developed to address these
challenges.  RCT are building blocks, which can be used as a stand-alone
system, or integrated with other RCT, or third-party tools to enable diverse
functionalities. RCT offer several innovative features to support the design
and implementation of middleware.

This paper takes a software perspective to present the overarching
architectural paradigm of RCT, discussing the design and implementation of
two cybertools: RADICAL-Pilot and Ensemble-Toolkit (EnTK). We outline the
direct impact that RCT are having on domain sciences, focusing the discussion
on architectural and design paradigms of middleware for scientific computing.
In this way, we show how RCT further the ``state-of-theory'' and practice of
scientific computing.


% ---------------------------------------------------------------------------
% Section II
% ---------------------------------------------------------------------------
\section{Software description}\label{sec:description}

% {\em Describe the software in as much as is necessary to establish a
% vocabulary needed to explain its impact.}

The RADICAL Cyberinfrastructure tools (RCT)~\cite{web-rct} have three main
components: RADICAL-SAGA (RS)~\cite{merzky2015saga}, RADICAL-Pilot
(RP)~\cite{merzky2018using} and RADICAL-Ensemble Toolkit
(EnTK)~\cite{balasubramanian2018harnessing}.

RS is a Python implementation of the Open Grid Forum SAGA standard 
GFD.90~\cite{goodale2006saga}, a high-level interface to distributed
infrastructure components like job schedulers, file transfer and resource
provisioning services. RS enables interoperability across heterogeneous
distributed infrastructures, improving on their usability and enhancing the
sustainability of services and tools.

RP is a Python implementation of the pilot paradigm and architectural
pattern~\cite{turilli2018comprehensive}. Pilot systems enable users to submit
pilot jobs to computing infrastructures and then use the resources acquired
by the pilot to execute one or more tasks. These tasks are directly scheduled
via the pilot, without having to queue in the infrastructure's batch system.
RP focuses on High Performance Computing (HPC) resources, enabling the
concurrent and consecutive execution of heterogeneous workloads comprised of
one or more scalar, MPI, OpenMP, multi-process, and multi-threaded tasks.
These tasks can be executed on CPUs, GPUs and other accelerators, on the same
pilot or across multiple pilots.

EnTK supports the concurrent or sequential execution of tasks that can be in
an arbitrary priority relation (i.e., ensemble or pipelines of tasks). EnTK
promotes ensembles of tasks to a high-level abstraction, providing a
programming interface and execution model specific to ensemble-based
applications. EnTK is engineered for scale and a diversity of computing
platforms and runtime systems, agnostic of the size, type and coupling of the
tasks comprising the ensemble.
\amnote{This is likely confusing if the reader does not know what
an 'ensemble' is?}\mtnote{better?}

RCT are designed to work both individually and as an integrated system, with
or without third party systems. This requires a ``Building Block'' approach
to their design and development, based on applying the traditional notions of
modularity at system level. The Building Block approach derives from the work
on Service-oriented Architecture and its Microservice variants, and the
component-based software development approaches where computational and
compositional elements are explicitly
separated~\cite{batory1992design,garlan1995architectural,lenz1988software,clemens1998component,schneider2000components}.
AirFlow, Oozie, Azkaban, Spark Streaming, Storm, or Kafka are examples of
tools that have a design consistent with the building blocks approach and
that have been integrated with RCT~\cite{}.

% In our adaptation, the Building Block approach is based on four
% well-understood principles: self-sufficiency, interoperability,
% composability, and extensibility. This approach does not reinvent
% modularity, it applies it at system level to enable composability among
% independent software systems. As an abstraction, modularity enables
% separation of concerns by encapsulating discrete functions into semantic
% units exposed by means of a dedicated interface. As such, modularity can be
% used at function or method level, depending on the programming paradigm and
% the facilities offered by programming languages, or at system-level,
% depending on the interface exposed by each system.

% ---------------------------------------------------------------------------
\subsection{Software Architecture}\label{ssec:architecture}

% {\em Give a short overview of the overall software architecture; provide a
% pictorial component overview or similar (if possible). If necessary provide
% implementation details.}

All RCT are stand-alone, distributed systems. Architecturally, each tool
consists of one or more subsystems, each with several components. Components
are isolated into individual processes and some components are used only in
specific deployment scenarios, depending on both application requirements and
resource capabilities. Components are stateless and some of them can be
instantiated concurrently to simultaneously manage multiple entities, like
workflows, workloads, tasks or pilots. This enables scaling of throughput and
tolerance to component failure.

Concurrent components are coordinated via a dedicated communication mesh,
which introduces runtime and infrastructure-specific overheads, but improves
overall scalability of the system and lowers component complexity. Components
can have different implementations; configuration files can tailor each RCT
to specific resources types, workloads, or scaling requirements. Components
exchange data about the entities specific to each RCT and data about the
state of the components and subsystems. Each type of data has dedicated
modules and communication channel, separating communication from coordination
with explicit states and events for each entity.
\amnote{what does 'making' mean here?}\mtnote{better?}

Ref.~\cite{merzky2015saga} details RADICAL-SAGA architecture and
capabilities. In the rest of the paper, we focus on RP and EnTK, first
introducing each system individually, then showing how RCT as a whole can be
composed to serve diverse use cases.

% ------------
\subsubsection{RADICAL-Pilot}\label{sssec:arch_rp}

RP implements two main abstractions: Pilot and Compute Unit (CU). Pilots and
CUs abstract away specificities of resources and workloads, making it
possible to schedule workloads either concurrently or sequentially on
resource placeholders. Pilots are such placeholders for computing resources,
where resources are represented independent from architecture and topological
details. CUs are units of work (i.e., tasks), specified as an application
executable alongside its resource and execution environment requirements.
Note that a task is not a function, method, thread or process but a program
that runs as a self-contained executable.

Fig.~\ref{fig:archs}a depicts RP's architecture with two subsystems (white
boxes) and several components (purple and yellow boxes). In each subsystem,
purple components manage pilots and CUs while yellow components manage the
communication among components. Subsystems can execute locally or remotely,
communicating and coordinating over TCP/IP, and enabling multiple deployment
scenarios. For example, users can run Client locally, and distribute MongoDB
and one or more instances of Agent on remote computing infrastructures.
Alternatively, users can run all components on a local or remote resource.

\begin{figure}
    \centering
    \subfloat[ ]{{\includegraphics[width=0.555\textwidth]{figures/arch_rp.pdf} }}
    \qquad
    \subfloat[ ]{{\includegraphics[width=0.34\textwidth]{figures/arch_entk.pdf} }}
    \caption{Caption. \mtnote{TODO\@: Fix EnTK AppManager}}\label{fig:archs}
\end{figure}

The first subsystem, called Client, has two main components: PilotManager and
UnitManager. PilotManager manages pilots and has a main component called
`Launcher'. Launcher uses resource configuration files to define the number,
placement, and properties of the Agent's components of each pilot. Currently,
configuration files are made available for all the HPC machines of the
Extreme Science and Engineering Discovery Environment (XSEDE), Blue Waters at
the National Center For supercomputing Applications (NCSA), Cheyenne at
NCAR-Wyoming Supercomputing Center (NWSC), and Rhea, Titan and Summit at the
Oak Ridge National Laboratory (ORNL). Users can provide new files or alter
existing configuration parameters at runtime, both for a single pilot or a
whole RP session.

UnitManager manages CUs and has two main components: Scheduler and
StagerInput. Scheduler schedules CUs onto one or more pilots, available on
one or more machines. This enables late binding of CUs to resources,
depending on their availability. CUs are bound to resources that satisfy
their execution requirements only when these resources are actually
available. StagerInput distributes the input files that CU may need for their
execution to the machines on which each CU has been scheduled.

The second subsystem of RP is called `Agent' and has four main components:
StagerInput and StagerOutput for staging CUs' input and output data,
Scheduler and Executer to schedule CUs on a pilot resources and execute those
CUs on them. Multiple instances of the Stager and Executer components can
coexist in a single Agent. Depending on the architecture of the target
machine, Agent's components can individually be placed on cluster head nodes,
MOM/batch nodes, compute nodes, virtual machines, or any combination thereof.
ZeroMQ communication bridges connect the Agent components, creating a network
to support the transitions of CUs through components.

Data management in RP focuses on providing input files to CUs before their
execution, and on moving output files to later CUs, or back into the user
environment. On HPC resources which provide both local and network storage,
RP can select the most appropriate storage, depending on CUs I/O
requirements.  Data can be exclusive to CUs, or can be shared between CUs.

Each component of each subsystem of RP has a dedicated queue to feed entities
into that component. Orange Queues in Fig.~\ref{fig:archs}a are dedicated to
pilots and CUs, blue Queues to the messages exchanged by communication
components. All Queues support bulk communication to obtain performance at
scale in a distributed systems. Further, queues enable load balancing among
concurrent components. Note that concurrent components are used for
performance optimization at scale.

A special queue instance is rendered as collection in a MongoDB database.
That collection is used to communicate between Client and Agent, while
preserving the semantics used for all other queues in Fig.~\ref{fig:archs}a.
Since the MongoDB entries are persistent, that database is also used to store
data for \textit{post mortem} profiling and analysis.

% ------------
\subsubsection{RADICAL Ensemble Toolkit}\label{sssec:arch_entk}

EnTK implements three main abstractions: task, stage and pipeline. Tasks
contain information regarding an executable, its software environment and its
data dependences. Stages are a set of tasks where the tasks have no mutual
dependences and can therefore execute concurrently, depending on resource
availability. Pipelines are lists of stages where a stage \(i\) can be
executed only after stage \(i-1\) has completed.

As for RP, a task is an executable, i.e., a program. \amnote{what is an
'instruction of a program'?}\mtnote{Not sure. Changed, alongside the rest of
the paragraph. Any better?} This is important: EnTK enables concurrent and
sequential execution at program-level, not at function or method level.
Parallelism is still possible within each program run by an EnTK task,
enabling concurrent execution of multi-threaded, multi-process and MPI
programs. Note that, at the moment, EnTK requires a runtime system (RTS) that
support the same task abstraction for task execution.

Fig.~\ref{fig:archs}b shows the architecture of EnTK (white) with three
components (purple), each with subcomponents dedicated to the management of
EnTK's entities (green) or coordination of the entities' execution (yellow).
The three components are AppManager, WFProcessor and TaskManager that enable
workflow specification, workflow execution management, and workload
management.

AppManager exposes an API for the development of ensemble-based applications
in terms of tasks, stages and pipelines, and for specifying resource
requirements for the application execution. AppManager initializes EnTK and
holds the global state of the application at runtime. AppManager is the sole
stateful component of EnTK, allowing to restart other components upon
failure, without interrupting the execution of the ensemble-based
application.

WFProcessor uses the Enqueue and Dequeue subcomponents to queue and dequeue
tasks pulled from AppManager. TaskManager uses ExecManager to schedule tasks
on the RTS and keep track of the state of each task during execution.
TaskManager uses ResManageer as an interface to the chosen RTS. RTS have to
provide capabilities to acquire resources and schedule tasks on those
resources for execution. ResManager isolates RTS from EnTK, enabling
restarting of the RTS without loosing information about tasks that have been
already executed. Currently, EnTK support only RP as RTS but it is designed
to use other task-based RTS as, for example, Coasters or HTCondor.

% ---------------------------------------------------------------------------
\subsection{Software Functionalities}\label{ssec:functionalities}

% {\em Present the major functionalities of the software.}

% SAGA Given the heterogeneity of distributed infrastructures, SAGA provides
% a much needed interoperability layer that lowers the complexity and
% improves the simplicity of using distributed infrastructure whilst
% enhancing the sustainability of distributed applications, services and
% tools.\mtnote{probably not to be included as SAGA has been already
% described in a standalone SoftwareX publication.}\amnote{agree}

% RP
As a pilot system, the defining capability of a RP is to decouple resource
acquisition from task execution, enabling execution of tasks on a pilot
without using the resource's queuing system. RP offers an API to describe
both pilots and CUs, alongside classes and methods to manage acquisition of
resources, scheduling and execution of CUs on those resources, and the
staging of input and output files. Reporting capabilities and notifications
update the user about ongoing executions, and profiling capabilities enable
detailed postmortem analysis of workload executions and runtime behavior.

% RP enables the execution of tasks with several dimensions of heterogeneity
% on one or more pilots instantiated on one or more resources. The defining
% capability of a pilot system is to decouple resource acquisition from task
% execution. Infrastructures exposing resources via a batch system allow
% users to submit their tasks as jobs. Jobs wait in the batch system's queue
% and when the requested resources became available for the requested amount
% of time, a job is scheduled for execution. Thus, each task of a many-task
% application has to wait in the batch system's queue to be executed,
% incurring in delays that can become rapidly unfeasible.

% Pilot systems allow for queuing a single job via the batch system and, once
% this job becomes active, it executes a system application that enables the
% direct scheduling of tasks on the acquired resources, without waiting in
% the batch system's queue. In this way, pilot systems can enable high
% throughput computing (HTC) on infrastructures designed to enable high
% performance computing (HPC). Note that pilot systems do not game the HPC
% infrastructure: pilot jobs are bound by the required resources and wait in
% the queue as every other job, therefore abiding to the system usage
% policies.

RP offers four unique features when compared to other pilot systems or tools
that enable the execution of many-task applications on HPC systems: (1)
concurrent execution of heterogeneous tasks on the same pilot; (2) support of
all the major HPC batch systems; (3) support of more than twelve methods to
launch tasks; and (4) a general purpose architecture. RP can execute single
or multi core tasks within a single compute node, or across multiple nodes,
isolating the execution of each tasks into a dedicated process and enabling
concurrent execution of heterogeneous tasks by design.

% EnTK EnTK is a workflow engine specifically designed to execute
% ensemble-based applications. These applications are composed by ensemble of
% tasks that may have dependences both among tasks and ensembles.\amnote{ah,
% here is the ensemble definition :-)}\mtnote{Right. see my previous editing,
% it should be consistent with this def.}

As a workflow engine, EnTK is designed to execute ensemble applications,
respecting the relations of priority among tasks. Compared to similar
systems, EnTK allows to codify relations in terms of pipelines, stages and
tasks, where relations may be determined by input/output data or control flow
requirements. For example, two tasks may have to be executed sequentially
when the output of the first task is the input of the second task; and two
tasks (or ensembles) may have to be executed sequentially when the output of
the first tasks determines whether executing the second task.

Consistently, EnTK provides adaptive capabilities and dedicated constructs to
pause, resume and stop pipelines at runtime. Adaptive applications change the
ensemble specifications, creating new pipelines, stages and tasks, or
changing the properties of those already defined. Further, pipelines and
stages can be paused while waiting to perform \textit{ad hoc} computations.
This enables the implementation of high-level application patterns as, for
example, simulation-analysis or replica exchange.

% Execution Model

% \mtnote{Describe BB and individual + aggregated execution models}


% ---------------------------------------------------------------------------
% \subsection{Sample code snippets analysis (optional)}\label{ssec:code}


% ---------------------------------------------------------------------------
% Section III
% ---------------------------------------------------------------------------
\section{Illustrative Examples}\label{sec:examples}

% {\em Provide at least one illustrative example to demonstrate the major
% functions.

% Optional: you may include one explanatory video that will appear next to
% your article, in the right hand side panel. (Please upload any video as a
% single supplementary file with your article. Only one MP4 formatted, with
% 50MB maximum size, video is possible per article. Recommended video
% dimensions are 640 x 480 at a maximum of 30 frames/second. Prior to
% submission please test and validate your .mp4 file at $
% http://elsevier-apps.sciverse.com/GadgetVideoPodcastPlayerWeb/verification$.
% This tool will display your video exactly in the same way as it will appear
% on ScienceDirect.).}

% \mtnote{Use a use case that requires a bag of heterogeneous tasks. Show how
% this example can become one state of a workflow and how this workflow is
% supported by EnTK\@. REPEX, EXTASY candidate examples.}

Multiple scientific domains can benefit from executing many-task applications
at scale, especially at the scale enabled by leadership-class HPC
machines~\cite{raicu2008many,iosup2011performance}. Independent of the domain
for which these applications are developed, their execution requires to run a
single task, a bag of tasks, or a workflow. In this context, tasks refers to
programs like, for example, GROMACS, NAMD, AMBER, AthenaMP, SPECFEM and many
others. Many-task applications requires to concurrently run multiple
instances of programs, using scale to reduce the total time to completion of
the whole execution.

As seen in Sec.~\ref{sec:description}, RCT support the execution of a single
task, a bag of tasks, and workflows expressed as a set or a sequence of
pipelines with stages and tasks. Because of the separation between manging
the concurrent and consecutive execution of tasks, and the computation
performance by each task, RCT support many-task application independent from
the scientific domain in which they are used. From RCT point of view, every
execution reduces exclusively to manging the execution of single or multiple
sets of programs in the form of black boxes.

RP executes set of tasks. The degree of concurrency of the execution depends
on the amount of available resources. Consider for example a many-task
application for the simulation of molecular dynamics with an ensemble of 128
simulations, each requiring 24 CPU cores as those used in
Ref.~\cite{balasubramanian2016extasy}. The user can use RP API to describe a
pilot job with 3072 cores (Code~\ref{code:pilot}), 128 tasks
(Code~\ref{code:units}) and two managers to coordinate the acquisition of the
pilot resources via RADICAL-SAGA on an HPC machine (Code~\ref{code:pmgr}) and
the execution of the 128 tasks on those resources (Code~\ref{code:umgr}).

Fig.~\ref{fig:archs}a's numbers illustrate the resource acquisition and task
execution process. PilotManager queues the pilot description on one of the
available Launcher in RP client (Fig.~\ref{fig:archs}a.1). That Launcher uses
RADICAL-SAGA to schedule the pilot as a job on the target resource via the
resource's batch System (Fig.~\ref{fig:archs}a.2). The pilot job waits in the
resource management system queue and, once scheduled, bootstraps the pilot's
AgentMnager and Updater. AgentManager forks the StagerInput, Scheduler,
Executor and StagerOutput components and the Updater notifies RP Client's
Notifier that RP Agent is ready to execute tasks (Fig.~\ref{fig:archs}a.3).

Upon notification, UnitManager queues all the available tasks onto Client's
Scheduler that, in turns, queues those tasks into the StagerInput, depending
on the chosen scheduling algorithm (Fig.~\ref{fig:archs}a.4). If required,
StagerInput stages the tasks' input files to the target resource and then
tasks are queued to the Updater and passed to the chosen RP Agent's
AgentManager (Fig.~\ref{fig:archs}a.5). At that point, tasks are passed to a
StagerInput where input files are linked and made available to each task, and
then queued to the RP Agent's Scheduler (Fig.~\ref{fig:archs}a.6). Scheduler
places tasks on suitable partitions of the pilot's resources and then queues
tasks to the Executor so that, when those partitions of resource becomes
available, tasks are executed (Fig.~\ref{fig:archs}a.7). Executor sets up the
environment required by each task and then forks each task for execution
(Fig.~\ref{fig:archs}a.8). This is why tasks are black boxes to RP\@; also
note that Scheduler and Executor can place and fork heterogeneous tasks,
i.e., task requiring different type and amount core/GPUs and different
execution time.

RP API cannot express dependences among tasks. For RP, every task that is
passed to UnitManager is assumed to be ready for execution. For example,
assume a typical simulation-analysis workflow for molecular dynamics with a
simulation stage and an analysis stage that depends upon the completion of
the simulation stage. Users can explicitly code priorities among stages in
the applications they write with the RP API but they have no dedicated
abstractions in that API for expressing those priorities. EnTK offers these
abstractions at API level: each stage of each pipeline is submitted to RP for
execution, respecting their priority relation.

Fig.~\ref{fig:archs}b's numbers illustrate the execution of workflows in
EnTK\@. Users instantiate an AppManager (Code~\ref{code:appnam}), define a
set of resources on which to run their workflow (Code~\ref{code:entk-res}),
describe that workflow in terms of pipelines, stages and tasks
(Code~\ref{code:pst}) and execute it (Code~\ref{code:pst}). AppManager passes
a copy of the workflow description to WFProcessor that, based on the
priorities between stages and tasks, uses Enquerer to queue tasks that are
ready for execution to the task manager (Fig.~\ref{fig:archs}b.1--2).
Meanwhile, ResManager users the chosen runtime system to acquire the
requested resources (Fig.~\ref{fig:archs}b.3) and, once available,
TaskManagers uses those resources to execute the queued tasks
(Fig.~\ref{fig:archs}b.4). ExecManager uses queues to communicate the state
of each task execution to AppManager. Based on this information, AppManager
copies tasks to WFProcessors when tasks dependences are satisfied. Note that
AppManager is the only stateful component of EnTK\@: both WFProcessor and
ExecManager can fail without loss of information about the
execution.\mtnote{Vivek to expand upon the dequeuer.}

% ---------------------------------------------------------------------------
% Section IV
% ---------------------------------------------------------------------------
\section{Impact}\label{sec:impact}

% {\em \textbf{This is the main section of the article and the reviewers
% weight the description here appropriately}

% Guidelines for the authors:
% \begin{enumerate}
%   \item Indicate in what way new research questions can be pursued as a
%         result of the software (if any).
%   \item Indicate in what way, and to what extent, the pursuit of existing
%         research questions is improved (if so).
%   \item Indicate in what way the software has changed the daily practice of
%         its users (if so).
%   \item Indicate how widespread the use of the software is within and
%         outside the intended user group.
%   \item Indicate in what way the software is used in commercial settings
%         and/or how it led to the creation of spin-off companies (if so).
% \end{enumerate}}

The impact of RCT spans domain science, high-performance computing and the
design of software systems. RCT have enabled domain-scientists to achieve
scientific results that would not have been possible otherwise; they have
facilitated research advances in high-performance and distributed computing
systems, while serving as an leading and important prototype implementation
for exploring a paradigmatic shift in the design of middleware for
high-performance scientific workflows.

RCT has enabled the development of scientific applications in multiple and
diverse domains, including software engineering, chemical physics, materials
science, health science, climate science, drug discovery and particle
physics. These users form a worldwide community of domain scientists and
system engineers that actively contribute to the open source development of
RCT. A comprehensive assessment across multiple dimensions is needed to
evaluate the true impact of a software system such as RCT. Whereas the
absolute number of users is a useful metric, an equally important metric is
what those users were able to achieve scientifically and how RCT enabled
them.

Currently, RCT supports a dozen active science projects across the USA and
Europe. The size of projects varies from single PIs with large allocations,
to very large international collaborations. Thus, there is intrinsic
uncertainty in the number of users at any given instant of time but good
faith, best estimates suggest upwards of 30 direct users.

RADICAL-SAGA and RADICAL-Pilot support use cases, spanning functional and
scientific domains. RADICAL-SAGA is mostly integrated into end-to-end
middleware solutions while RADICAL-Pilot is used both as standalone system
and integrated with other systems. As seen in Sec.~\ref{ssec:architecture},
RADICAL-PILOT uses RADICAL-SAGA to submit pilots to a large array of
resources, including HPC and distributed systems.

RADICAL-SAGA enables the Production ANd Distributed Analysis (PanDA) system
to submit batch jobs to Titan and Summit, the two leadership class machines
managed by the Oak Ridge Leadership Computing Facility (OLCF) at the Oak
Ridge National Laboratory (ORNL)~\cite{}. PanDA is the workload management
system used by the ATLAS experiment to execute hundred of millions of jobs a
year on both grid and High Performance Computing (HPC)
infrastructures~\cite{}. The usage of ORNL resources constitutes 10-12\% of
all of ATLAS computing. There are several thousand researchers that directly
or indirectly use PanDA, and thereby RADICAL-SAGA. In the near future,
RADICAL-Pilot will also become a staple of the PanDA workload management
system on HPC platforms.

Reflecting the state of distributed computing systems -- the lack of
simplified and uniform interface to heterogeneous systems, RADICAL-SAGA was
used to develop Science Gateways as part of the Distributed Application
Runtime Environment (DARE) framework. These gateways supported several
projects, including DECIDE and neuGRID, to study the early diagnosis of
Alzheimer and other neurodegenerative diseases. In that capacity RADICAL-SAGA
enabled submission of jobs to distributed computing infrastructures managed
by the European Grid Initiative (EGI), interconnected via GEANT, the
pan-European research and education network that interconnects Europe’s
National Research and Education Networks. Recently, the emergence of toolkits
such as Agave which integrate identity management have provided higher-level
solutions for Gateway developers, but they retain the RADICAL-SAGA based
approach to job submission to distributed computing systems.

Since its first release in 2013, RP has supported a total of two dozen
projects and around 100 active and direct users. Of these, approximately a
dozen projects used RP has a standalone system to support the execution of
many-task applications on single and/or multiple computing infrastructures.
Motivated by the practical lessons from supporting many independent
applications usage of RP as a standalone system, and the realization that an
increasing number of HPC applications were adopting the ensemble
computational model to overcome limitations of single task applications to
achieve significant performance gains on large-scale parallel machines, in
2015 we designed and implemented the Ensemble Toolkit
(EnTK)~\citep{balasubramanian2016extasy} as the latest addition to RCT.

EnTK has enabled the development of domain specific workflow (DSW) frameworks
which provide a specific higher-level functionality. Although, driven by
specific application needs, each DSW is characterized by a unique execution
and coordination pattern and can serve multiple applications. The four
ensemble-based DSW developed using EnTK and other RCT are:
EXTASY~\cite{balasubraman,ian2016extasy}, RepEx~\cite{treikalis2016repex},
HTBAC~\cite{dakka2018high}, and ICEBERG~\cite{}. Details can be found in
Ref.~\citep{}  % reference CISE article.

ExTASY and RepEx implement advanced sampling algorithms using biomolecular
simulations. Both use the EnTK API to implement  diverse coordination
patterns amongst ensembles of biomolecular simulations and analysis. HTBAC
supports multiple algorithms that compute free-energy calculations that are
critical to drug design and resistance studies. HTBAC allows the runtime
adaptation at multiple levels: algorithms, pipelines and tasks within a
pipeline. This capability has been demonstrated to reduce the
time-to-solution by a factor 2.5 in controlled experiments on real drug
candidates~\citep{}. ICEBERG supports scalable image analysis applications
using multiple concurrent pipelines.

ExTASY, RepEx, HTBAC and ICEBERG benefit from integrating RCT by not having
to reimplement workflow processing, efficient task management and
interoperable task execution capabilities on distinct and heterogeneous
platforms. This, in turn, enables both a focus on and ease of ``last mile
customization'' for the DSW\@.

RCT are a testbed for engineering research, mostly focused on foundational
abstractions~\cite{}, architectural paradigms~\cite{}, application
patterns~\cite{}, and performance analysis of distributed middleware on
diverse computing infrastructures~\cite{}. Among the most representative
projects supported by RP as a standalone system, the Abstractions and
Integrated Middleware for Extreme-Scale Science (AIMES) project enabled
extreme-scale distributed computing via dynamic federation of heterogeneous
computing infrastructures. We used RP to execute millions of tasks on both
HPC and HTC resources, studying the federated behavior of multiple
infrastructures, establishing for the first time ever the importance of
integrating task and resource information in scheduling and placement
decision making for federated supercomputers~\cite{}.

RCT is a small operation with at most two developers working on the systems
at the same time. In order to implement the aforementioned capabilities while
supporting more than ten concurrent projects at every point in time, we
adopted a specific methodology for the design, development and maintenance of
RCT. This methodology is based on the building block approach, defined in
Ref.~\cite{bb} and briefly outlined in Sec.~\ref{}, the use of git for
distributed version control of the code base, and a tailored project
management process.

The building blocks approach helps to create systems that can support use
cases both individually and as integrated, end-to-end solution. This is
important when supporting projects with multiple, distinct use cases. For
example, ICEBERG has to support five use cases, each investigating a specific
problem in the domain of polar science. Four of these use cases require the
concurrent execution of pipelines but one requires only the execution of bag
of tasks. The first four cases can use EnTK while the latter only
RADICAL-Pilot. Importantly, all use cases can be served by the ICEBERG
framework with a minor change in the private API to call EnTK or RP,
depending on the use cases and therefore without any engineering overheads.

The building block approach represents an important advance in the design of
middleware for scientific applications. The building blocks approach,
leverages emerging trends in software and distributed computing
infrastructure, to enable a sustainable ecosystem of both existing and new
software components from which tailored workflow systems can be composed.
Building blocks enable expert contributions while lowering the breadth of
expertise required of workflow system developers. Building blocks provide
both a technical basis and the socio-economic incentives to support an
integrative and sustainable approach to the design of workflow systems.

Git enables the project to benefit from an distributed open source
development model. Code is publicly available and everyone can contribute new
code by opening a pull request. This has the double benefit of helping to
build a global community around RCT but also to facilitate the integration of
new lead developers. As RCT are developed in an academic context, changes in
available development effort alongside the technical quality of that effort
need to be accounted for in the development model. To this end, documentation
is managed via the wiki functionality of GitHub, Sphinx and Read the Docs.
Embedding documentation in the code allows to facilitate the adoption of the
code base while maintaining high quality, up to date documentation. Wiki
enables distributed editing of high-level, design and project-related
documentation.

The project management process of RCT is based on the notion of `action'.
Each action is assigned to one or multiple owners and reviewed in weekly
meetings. Actions are implemented as GitHub tickets, facilitating integration
with pull requests and wiki-based documentation. Tickets labeling uses a
predefined taxonomy, shared among the RCT repositories. Tickets' labels and
metadata are pulled regularly into a database to support statistical
analysis. The insight is used to determine code hot spots, priority among
feature requests and critical user issues. In turn, this enables allocation
of development effort and planning for upcoming feature development and
releases.

Overall, these processes and insight have an impact on how to approach the
development of middleware for supporting scientific research. Based on more
than ten years of experience, our approach show a sustainable and effective
way to organize software development, promote community adoption and leverage
the specific characteristics of the academic financial model. This signs the
transition from a development model based on end-to-end, monolithic solutions
with stringent requirements on infrastructures' software stack, to a model
based on small, independent and composable systems, each with a well-defined
capability. Note that these systems must be composable with third-party
systems, i.e., systems developed independently by different development
teams. This approach enables a model of sustainability based on smaller and
shorter funding sources but requires a certain convergence in the vision of
diverse groups competing in the same research field.


% ---------------------------------------------------------------------------
% Section V
% ---------------------------------------------------------------------------
\section{Conclusions}\label{sec:conclusions}

{\em Set out the conclusion of this original software publication.}


% ---------------------------------------------------------------------------
% Acknowledgements
% ---------------------------------------------------------------------------
\section*{Acknowledgements}

{\em Optionally thank people and institutes you need to acknowledge. }


% ---------------------------------------------------------------------------
% References
% ---------------------------------------------------------------------------
\bibliographystyle{elsarticle-num} 
\bibliography{rct-softwarex}


% ---------------------------------------------------------------------------
% Appendixes
% ---------------------------------------------------------------------------
\appendix


% ---------------------------------------------------------------------------
% Appendix I
% ---------------------------------------------------------------------------
\section*{Required Metadata}\label{sec:metadata}


% ---------------------------------------------------------------------------
% Appendix II
% ---------------------------------------------------------------------------
\section*{Current code version}\label{sec:src_version}

% {\em Ancillary data table required for subversion of the codebase. Kindly
% replace examples in right column with the correct information about your
% current code, and leave the left column as it is.

\begin{table}[!ht]
\begin{tabular}{|l|p{6.5cm}|p{6.5cm}|}
\hline
\textbf{Nr.}                                                     & 
\textbf{Code metadata description}                               & 
\textbf{Please fill in this column}                              \\
\hline
%
C1                                                               & 
Current code version                                             & 
0.60                                                             \\
\hline
%
C2                                                               & 
Permanent link to code/repository used for this code version     & 
\url{https://github.com/radical-cybertools}                      \\
\hline
%
C3                                                               & 
Legal Code License                                               & 
MIT License (MIT)                                                \\
\hline
%
C4                                                               & 
Code versioning system used                                      & 
git                                                              \\
\hline
%
C5                                                               & 
Software code languages, tools, and services used                & 
python, shell, C                                                 \\
\hline
%
C6                                                               & 
Compilation requirements, operating environments \& dependencies & 
virtualenv, pip or conda                                         \\
\hline
%
C7                                                               & 
Developer documentation/manual                                   & 
\url{https://radicalpilot.readthedocs.io/}                       \\
\hline
%
C8                                                               & 
Support email for questions                                      & 
\url{radical-cybertools@googlegroups.com}                        \\
\hline
\end{tabular}
\caption{Code metadata}\label{tab:src_metadata} 
\end{table}


% ---------------------------------------------------------------------------
% Appendix III
% ---------------------------------------------------------------------------
\section*{Current executable software version}\label{sec:bin_version}

% {\em Ancillary data table required for sub version of the executable
% software: (x.1, x.2 etc.) kindly replace examples in right column with the
% correct information about your executables, and leave the left column as it
% is.

\begin{table}[!ht]
\begin{tabular}{|l|p{6.5cm}|p{6.5cm}|}
\hline
\textbf{Nr.}                                                     & 
\textbf{(Executable) software metadata description}              & 
\textbf{Please fill in this column}                              \\
\hline
%
S1                                                               & 
Current software version                                         & 
0.60                                                             \\
\hline
%
S2                                                               & 
Permanent link to executables of this version                    & 
\url{https://pypi.org/project/radical.pilot/}                    \\
\hline
%
S3                                                               & 
Legal Software License                                           & 
MIT License (MIT)                                                \\
\hline
%
S4                                                               & 
Computing platforms/Operating Systems                            & 
GNU/Linux operating systems                                      \\
\hline
%
S5                                                               & 
Installation requirements \& dependencies                        & 
virtualenv, pip or conda                                         \\
\hline
%
S6                                                               & 
User manual and publications                                     & 
User manual: \url{https://radicalpilot.readthedocs.io/}; 
Publications: Refs~\cite{}                                       \\
\hline
%
S7                                                               & 
Support email for questions                                      & 
\url{radical-cybertools@googlegroups.com}                        \\
\hline
\end{tabular}
\caption{Software metadata}
\label{tab:bin_metadata} 
\end{table}

\end{document}
\endinput
