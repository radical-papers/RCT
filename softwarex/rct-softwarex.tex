\documentclass[preprint,12pt, a4paper]{elsarticle}

\usepackage{amsmath}
\usepackage{amssymb}
\usepackage{array}
\usepackage{booktabs}
\usepackage{color}
\usepackage{float}
\usepackage{graphicx}
\usepackage{ifpdf}
\usepackage[utf8]{inputenc}
\usepackage{keyval}
\usepackage{lineno}
\usepackage{listings}
\usepackage{longtable}
\usepackage{moresize}
\usepackage{multirow}
\usepackage{paralist}
\usepackage{rotating}
\usepackage{soul}
\usepackage{srcltx}
\usepackage{url}
\usepackage{xcolor}
\usepackage{xspace}
\usepackage{wrapfig}
\usepackage{subfig}
\usepackage{tikz}
\usepackage[export]{adjustbox} % http://ctan.org/pkg/adjustbox

\definecolor{listinggray}{gray}{0.95}
\definecolor{darkgray}{gray}{0.7}
\definecolor{commentgreen}{rgb}{0, 0.4, 0}
\definecolor{darkblue}{rgb}{0, 0, 0.6}
\definecolor{purple}{rgb}{0.6, 0, 0.6}
\definecolor{middleblue}{rgb}{0, 0, 0.75}
\definecolor{darkred}{rgb}{0.4, 0, 0}
\definecolor{brown}{rgb}{0.5, 0.5, 0}
\definecolor{dkgreen}{rgb}{0,0.5,0}
\definecolor{orange}{rgb}{1,.5,0}
\definecolor{dandelion}{cmyk}{0,0.29,0.84,0}

\usepackage[normalem]{ulem}
\makeatletter
\def\cyanuwave{\bgroup \markoverwith{\lower3.5\p@\hbox{\sixly \textcolor{cyan}{\char58}}}\ULon}
\def\reduwave{\bgroup \markoverwith{\lower3.5\p@\hbox{\sixly \textcolor{red}{\char58}}}\ULon}
\def\blueuwave{\bgroup \markoverwith{\lower3.5\p@\hbox{\sixly \textcolor{blue}{\char58}}}\ULon}
\font\sixly=lasy6 % does not re-load if already loaded, so no memory problem.
\makeatother

\def\BibTeX{{\rm B\kern-.05em{\sc i\kern-.025em b}\kern-.08em
    T\kern-.1667em\lower.7ex\hbox{E}\kern-.125emX}}

% Generate circled numbers
\newcommand*\circled[1]{\tikz[baseline=(char.base)]{
    \node[shape=circle,draw,inner sep=1pt] (char) {#1};}}
\newif\ifdraft{}
\drafttrue{}

\ifdraft{}
  \newcommand{\amnote}[1]{ \textcolor{blue} { ***andrem: #1 }}
  \newcommand{\jhanote}[1]{ {\textcolor{red} { ***shantenu: #1 }}}
  \newcommand{\mtnote}[1]{ {\textcolor{orange} { ***matteo: #1 }}}
\else
  \newcommand{\amnote}[1]{}
  \newcommand{\jhanote}[1]{}
  \newcommand{\mtnote}[1]{}
\fi

\newcommand{\apples}{AppLeS\xspace}
\newcommand{\bj}{BigJob\xspace}
\newcommand{\computeunit}{Compute-Unit\xspace}
\newcommand{\computeunits}{Compute-Units\xspace}
\newcommand{\cloud}{cloud\xspace}
\newcommand{\clouds}{clouds\xspace}
\newcommand{\cc}{c\&c\xspace}
\newcommand{\CC}{C\&C\xspace}
\newcommand{\computedataservice}{Compute-Data Service\xspace}
\newcommand{\cu}{CU\xspace}
\newcommand{\cus}{CUs\xspace}
\newcommand{\dataunit}{Data-Unit\xspace}
\newcommand{\dataunits}{Data-Units\xspace}
\newcommand{\du}{DU\xspace}
\newcommand{\dus}{DUs\xspace}
\newcommand{\mrmg}{MR-Manager\xspace}
\newcommand{\MW}{master-worker\xspace}
\newcommand{\numrep}{8 }
\newcommand{\panda}{PanDA\xspace}
\newcommand{\pilot}{Pilot\xspace}
\newcommand{\pilots}{Pilots\xspace}
\newcommand{\pilotjob}{Pilot-Job\xspace}
\newcommand{\pilotjobs}{Pilot-Jobs\xspace}
\newcommand{\pilotcompute}{Pilot-Compute\xspace}
\newcommand{\pilotcomputes}{Pilot-Computes\xspace}
\newcommand{\pilotdata}{Pilot-Data\xspace}
\newcommand{\pilotdataservice}{Pilot-Data Service\xspace}
\newcommand{\pilotcomputeservice}{Pilot-Compute Service\xspace}
\newcommand{\prop}[1]{\textit{#1}\xspace}
\newcommand{\pilotmapreduce}{PilotMapReduce\xspace}
\newcommand{\pstar}{P*\xspace}
\newcommand{\pd}{PD\xspace}
\newcommand{\pj}{PJ\xspace}
\newcommand{\pjs}{PJs\xspace}
\newcommand{\pds}{Pilot Data Service\xspace}
\newcommand{\samplenum}{4 }
\newcommand{\su}{SU\xspace}
\newcommand{\sus}{SUs\xspace}
\newcommand{\schedulableunit}{Schedulable Unit\xspace}
\newcommand{\schedulableunits}{Schedulable Units\xspace}
\newcommand{\tmax}{\(T_{max}\)}
\newcommand{\tc}{\(T_{C}\)}
\newcommand{\tcnsp}{\(T_{C}\)}
\newcommand{\vocab}[1]{\textbf{#1}\xspace}

\newcommand{\B}[1]{\textbf{#1}\xspace}
\newcommand{\C}[1]{\textsc{#1}\xspace}
\newcommand{\F}[1]{\textbf{FIXME\@: #1}\xspace}
\newcommand{\I}[1]{\textit{#1}\xspace}
\newcommand{\T}[1]{\texttt{#1}\xspace}

\newcommand{\impterm}[1]{\texttt{#1}\xspace}

% System names
\newcommand{\bw}{\I{Blue\,Waters}}
\newcommand{\stampede}{\I{Stampede}}
\newcommand{\comet}{\I{Comet}}
\newcommand{\titan}{\I{Titan}}

% Latex Fu
\newcommand{\UPP}{\vspace*{-2.0em}}
\newcommand{\UP}{\vspace*{-1.0em}}
\newcommand{\up}{\vspace*{-0.5em}}

% Paper specific Macro's
\newcommand{\ru}{$RU$\xspace}
\newcommand{\ttc}{$ttc$}
\newcommand{\ttca}{$ttc_a$}

% Table multirows
\newcommand{\mr}[1]{\multirow{2}{*}{#1}}%
\newcommand{\mc}[2]{\multicolumn{#1}{l}{#2}}

\lstdefinestyle{myListing}{
  frame=single,   
  backgroundcolor=\color{listinggray},  
  %float=t,
  language=C,       
  basicstyle=\ttfamily \footnotesize,
  breakautoindent=true,
  breaklines=true
  tabsize=2,
  captionpos=b,  
  aboveskip=0em,
  belowskip=-2em,
}      

\lstdefinestyle{myPythonListing}{
  frame=single,   
  backgroundcolor=\color{listinggray},  
  %float=t,
  language=Python,       
  basicstyle=\ttfamily \footnotesize,
  breakautoindent=true,
  breaklines=true
  tabsize=2,
  captionpos=b,  
}

% This is now the recommended way for checking for PDFLaTeX:
\ifpdf{}
  \DeclareGraphicsExtensions{.pdf, .jpg, .tif}
\else
  \DeclareGraphicsExtensions{.eps, .jpg, .ps}
\fi

\tolerance=1000
\hyphenpenalty=10


\journal{SoftwareX}

\begin{document}
\begin{frontmatter}

% \title{Title/Name of your software}
\title{RADICAL Cybertools}

\author{A. Author}
\address{Your institute, some address}

\author{B. Author}
\address{Your institute, some address}

\author{C. Author}
\address{Your institute, some address}

\author{D. Author}
\address{Your institute, some address}

\begin{abstract}
Ca. 100 words
\end{abstract}

\begin{keyword}
%% keywords here, in the form: keyword \sep keyword
keyword 1 \sep keyword 2 \sep keyword 3

%% PACS codes here, in the form: \PACS code \sep code

%% MSC codes here, in the form: \MSC code \sep code
%% or \MSC[2008] code \sep code (2000 is the default)

\end{keyword}

\end{frontmatter}

\linenumbers

% The manuscript must be submitted in single column. The following
% constraints apply: 
% Word count: max. 3000.
% a.	Excluding: title, authors, affiliations, references, metadata tables.
% b.	Including: abstract, running text, captions, footnotes.
% c.	Max. 6 figures.


% ---------------------------------------------------------------------------
% Section I
% ---------------------------------------------------------------------------
\section{Motivation and significance}\label{sec:motivation}

Introduce the scientific background and the motivation for developing the
software.

Explain why the software is important, and describe the exact (scientific)
problem(s) it solves.

Indicate in what way the software has contributed (or how it will contribute
in the future) to the process of scientific discovery; if available, this is
to be supported by citing a research paper using the software.

Provide a description of the experimental setting (how does the user use the
software?).

Introduce related work in literature (cite or list algorithms used, other
software etc.).


% ---------------------------------------------------------------------------
% Section II
% ---------------------------------------------------------------------------
\section{Software description}\label{sec:description}

Describe the software in as much as is necessary to establish a vocabulary
needed to explain its impact.


% ---------------------------------------------------------------------------
% Section III
% ---------------------------------------------------------------------------
\subsection{Software Architecture}\label{sec:architecture}

Give a short overview of the overall software architecture; provide a
pictorial component overview or similar (if possible). If necessary provide
implementation details.


% ---------------------------------------------------------------------------
% Section IV
% ---------------------------------------------------------------------------
\subsection{Software Functionalities}\label{sec:functionalities}

Present the major functionalities of the software.


% ---------------------------------------------------------------------------
% Section V
% ---------------------------------------------------------------------------
\subsection{Sample code snippets analysis (optional)}\label{sec:code}


% ---------------------------------------------------------------------------
% Section VI
% ---------------------------------------------------------------------------
\section{Illustrative Examples}\label{sec:examples}

Provide at least one illustrative example to demonstrate the major functions.

Optional: you may include one explanatory video that will appear next to your
article, in the right hand side panel. (Please upload any video as a single
supplementary file with your article. Only one MP4 formatted, with 50MB
maximum size, video is possible per article. Recommended video dimensions are
640 x 480 at a maximum of 30 frames/second. Prior to submission please test
and validate your .mp4 file at $
http://elsevier-apps.sciverse.com/GadgetVideoPodcastPlayerWeb/verification$.
This tool will display your video exactly in the same way as it will appear
on ScienceDirect.).


% ---------------------------------------------------------------------------
% Section VII
% ---------------------------------------------------------------------------
\section{Impact}\label{sec:impact}

\textbf{This is the main section of the article and the reviewers weight the
description here appropriately}

Indicate in what way new research questions can be pursued as a result of the
software (if any).

Indicate in what way, and to what extent, the pursuit of existing research
questions is improved (if so).

Indicate in what way the software has changed the daily practice of its users
(if so).

Indicate how widespread the use of the software is within and outside the
intended user group.

Indicate in what way the software is used in commercial settings and/or how
it led to the creation of spin-off companies (if so).


% ---------------------------------------------------------------------------
% Section VIII
% ---------------------------------------------------------------------------
\section{Conclusions}\label{sec:conclusions}

Set out the conclusion of this original software publication.


% ---------------------------------------------------------------------------
% Acknowledgements
% ---------------------------------------------------------------------------
\section*{Acknowledgements}

Optionally thank people and institutes you need to acknowledge. 

%% The Appendices part is started with the command \appendix;
%% appendix sections are then done as normal sections
%% \appendix

%% \section{}
%% \label{}


% ---------------------------------------------------------------------------
% References
% ---------------------------------------------------------------------------
\bibliographystyle{elsarticle-num} 
\bibliography{rct-software}


% ---------------------------------------------------------------------------
% Appendix I
% ---------------------------------------------------------------------------
\section*{Required Metadata}\label{sec:metadata}


% ---------------------------------------------------------------------------
% Appendix II
% ---------------------------------------------------------------------------
\section*{Current code version}\label{sec:src_version}

Ancillary data table required for subversion of the codebase. Kindly replace
examples in right column with the correct information about your current
code, and leave the left column as it is.

\begin{table}[!ht]
\begin{tabular}{|l|p{6.5cm}|p{6.5cm}|}
\hline
\textbf{Nr.}                                                     & 
\textbf{Code metadata description}                               & 
\textbf{Please fill in this column}                              \\
\hline
C1                                                               & 
Current code version                                             & 
For example v42                                                  \\
\hline
C2                                                               & 
Permanent link to code/repository used for this code version     & 
For example: $https://github.com/mozart/mozart2$                 \\
\hline
C3                                                               & 
Legal Code License                                               & 
List one of the approved licenses                                \\
\hline
C4                                                               & 
Code versioning system used                                      & 
For example svn, git, mercurial, etc. put none if none           \\
\hline
C5                                                               & 
Software code languages, tools, and services used                & 
For example C++, python, r, MPI, OpenCL, etc.                    \\
\hline
C6                                                               & 
Compilation requirements, operating environments \& dependencies & 
                                                                 \\
\hline
C7                                                               & 
If available Link to developer documentation/manual              & 
For example: $http://mozart.github.io/documentation/$            \\
\hline
C8                                                               & 
Support email for questions                                      & 
                                                                 \\
\hline
\end{tabular}
\caption{Code metadata (mandatory)}\label{tab:src_metadata} 
\end{table}


% ---------------------------------------------------------------------------
% Appendix III
% ---------------------------------------------------------------------------
\section*{Current executable software version}\label{sec:bin_version}

Ancillary data table required for sub version of the executable software:
(x.1, x.2 etc.) kindly replace examples in right column with the correct
information about your executables, and leave the left column as it is.

\begin{table}[!ht]
\begin{tabular}{|l|p{6.5cm}|p{6.5cm}|}
\hline
\textbf{Nr.} & 
\textbf{(Executable) software metadata description} & 
\textbf{Please fill in this column} \\
\hline
S1 & 
Current software version & 
For example 1.1, 2.4 etc. \\
\hline
S2 & 
Permanent link to executables of this version  & 
For example: $https://github.com/combogenomics/$ $DuctApe/releases/tag/DuctApe-0.16.4$ 
\\
\hline
S3 & 
Legal Software License & 
List one of the approved licenses \\
\hline
S4 & 
Computing platforms/Operating Systems & 
For example Android, BSD, iOS, Linux, OS X, Microsoft Windows, Unix-like , IBM z/OS, distributed/web based etc. \\
\hline
S5 & 
Installation requirements \& dependencies & 
\\
\hline
S6 & 
If available, link to user manual - if formally published include a reference to the publication in the reference list & 
For example: $http://mozart.github.io/documentation/$ \\
\hline
S7 & 
Support email for questions & 
\\
\hline
\end{tabular}
\caption{Software metadata (optional)}
\label{tab:bin_metadata} 
\end{table}

\end{document}
\endinput